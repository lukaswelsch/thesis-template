\chapter{Daten}\label{ch:data}

Die in der Arbeit verwendeten Daten stammen aus dem Data Warehouse einer deutschen Bank.
Aufgrund des Datenschutzes wurden personenbezogene Daten pseudonymisiert und anonymisiert.

Ein- und Ausschlusskriterien für die Daten:



grobe Anzahl des Datenbestandes (evtl. auf dem produktiv System) berechnen


Risikoscoring (Werte von 1A bis 4E)
Trainigslabel = Scoring


 Name (Vorname), Alter, Geschlecht, Familienstand, Anzahl der Kinder,
Alter der Kinder, Meldeadresse(n), Wohndauer, Haushaltstyp, Bildungsstand, Beruf, Arbeitgeber, Beschäftigungsdauer, monatliches Nettoeinkommen, monatliche Ausgaben (Haushaltsrechnung), Kfz-Besitz, Eintragungen in Schuldnerverzeichnissen und Warnlisten, Insolvenz, gebotene Sicherheiten (z. B. Immobilien, Bürgen), Kontoführung und Überziehungen, Dauer der Kundenbeziehung, auffällige Einzeltransaktionen, vorherige interne
Kredite und Erfahrungen hieraus, sowie Art und Anzahl der Kredite. 





Im vorliegenden Projekt wird zunächst mit den Entwicklungsdaten gearbeitet, da diese im wesentlichen den produktiven Daten entsprechen.
Der Vorteil an Entwicklungsdaten liegt darin, dass mit diesen Daten Experimente durchgeführt werden können, ohne den Datenschutz zu missachten, da diese schon pseudonymisiert und anonymisiert worden sind. 

Zunächst sollte sich eine Übersicht über die Daten beschafft werden:


\begin{lstlisting}[language=SQL,caption={Überblick über die Daten},captionpos=b]
SELECT VALID_DT, COUNT(VALID_DT) 
FROM SCHEMA.STAMMDATEN 
GROUP BY VALID_DT 
ORDER BY VALID_DT
\end{lstlisting}
Anhand dieser Abfrage lassen sich die Anzahl der Daten zu den jeweiligen Tagen bestimmen. 
Beispielhaft wird mit einem aktuellen VALID_DT gearbeitet. 
Folgende Spalten sind in der Tabelle vorhanden:






Visualisierung auf zwei Arten: Major critical (Sonar)
1. Visualisierung der Metriken (Ergebnisse)
2. Visualisuerng der Daten selbst, zb Nul Values gelb markiert
