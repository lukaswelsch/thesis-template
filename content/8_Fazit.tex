\chapter{Fazit}\label{ch:summary}

Datenqualität ist unsichtbar, wenn alles richtig gemacht wird.

Es ist schwer zum Thema Datenqualität konkrete Lösungen bzw. Anwendungsfälle zu finden, da die meisten Arbeiten nur auf einem theoretischem Niveau unterwegs sind.
Auch große Beratungsfirmen zeigen nur Schritte auf, wie es theoretisch möglich ist, die Datenqualität zu Monitoren und zu verbessern.

Diese Arbeit zeigt exemplarisch für einen konkreten Fall, wie für diesen die Datenqualität im Bezug auf Aktualität analysiert und auch verbessert werden kann.

Allerdings ist es nicht für alle Problemfelder möglich mit automatischen Verfahren zu arbeiten, sondern es ist das Wissen von Businessusern bzw. Stakeholdern nötig, die konkrete Zielwerte vorgeben, um die Datenqualität zu überprüfen.
Hierfür sollten außer den vorgestellten Methoden noch manuelle Verfahren etabliert werden, die überprüfen, ob bestimmte Zielwerte eingehalten werden. 
Dafür sind umfangreiche Aussagen über die Daten nötig, wie z. B. Schwellwerte. 
Die benötigten Metadaten müssten zunächst von Experten festgelegt werden.

Des weiteren könnte ein Datensatz hilfreich sein, der konkrete Labels für gute bzw. schlechte Datenqualität besitzt. 
Dies könnte erreicht werden, indem Änderungen am Datensatz nur mit einer Begründung erfolgen kann.
Der Bearbeiter, der den Datensatz ändert, müsste so hinterlegen, warum der Datensatz aktualisiert worden ist. 

Da dieser Datensatz nicht existiert, musste damit gearbeitet werden, was vorhanden ist.
