\chapter{Einleitung}\label{ch:intro}

%You may have read about similar things in \cite{Goodliffe2007}.
%You can also write footnotes.\footnote{Footnotes will be positioned automatically.}
%It is possible to reference glossary entries as \gls{library} as an example.




- Kann mit Hilfe von ML-Verfahren berechnet werden, ob Risikobewertungen über Kunden aktualisiert werden müssen? 
% externe Dienstleister bündeln die Informationen mehrerer Kreditvergeber. 
% Deshalb werden die Daten von dort vermutlich immer akurater sein und können nicht ersetzt werden
- Können die vorliegenden MetaDaten visualisiert werden, sodass Erkenntnisse zur (Prozess)-Qualität gewonnen werden können? 

\cite{d.kelleher2015a}
\cite{d.kelleher2015}

%IDEEN
%versch. Arten visualisieren und dann selbst entscheiden bzw Entscheidungen von Experten mit einfließen lassen
%Anpassung der Adressen aus den letzten Jahren (Durchschnitt wie oft wird Kundenstammdaten aktualisiert)



% Problematisch ist, dass die IT-Systeme so gestaltet sind, dass diese Freitextfelder verfügbar sind, sodass erst keine Rechtschreibfehler entstehen können. 
% Allerdings steht die Bank vor dem Problem die Daten schon erhoben zu haben und benötigt jetzt eine Lösung, um diese auszubessern. 

