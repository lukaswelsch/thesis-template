\chapter{Einleitung}\label{ch:intro}

%You may have read about similar things in \cite{Goodliffe2007}.
You can also write footnotes.\footnote{Footnotes will be positioned automatically.}

Wie kann die Daten Qualität aufbereitet und visualisiert werden, sodass Entscheider schnell und gezielt Maßnahmen einleiten können?
-> Wie visualisiert man am besten?
-> Wie kann man die DQ messen? (Am besten ohne großen Aufwand durch die Experten, also möglichst viel automatisch machen)
-> 

versch. Arten visualisieren und dann selbst entscheiden bzw Entscheidungen von Experten mit einfließen lassen
Anpassung der Adressen aus den letzten Jahren (Durchschnitt wie oft wird Kundenstammdaten aktualisiert)

\section{This is an Important Section}

It is possible to reference glossary entries as \gls{library} as an example.


\subsection{And an even more important subsection}
