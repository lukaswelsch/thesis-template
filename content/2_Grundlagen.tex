\chapter{Grundlagen}\label{ch:data}

\section{Definition von Datenqualität}
Datenqualität wird in der Literatur auf verschiedene Arten definiert. 

\subsection{Datenqualitätsdimensionen}
In der Literatur werden als Basisset der Datenqualitätsdimensionen Richtigkeit, Vollständigkeit, Konsistenz und Aktualität definiert. \cite{scannapieco2002}

\subsection{Definition / Metriken der Dimensionen}
Um diese eher abstrakten Definitionen messbar zu machen haben sich einige Verfahren etabliert.
Diese lassen sich grob in die Kategorien subjektiv und objektiv aufteilen. \cite{pipino2002}
Bei den subjektiven Verfahren werden die Grenzwerte, ab wann die Datenqualität schlecht ist durch Experten geschätzt. 
Die Experten geben oft auch ein Gefühl ab, ob sie die Datenqualität für gut oder schlecht halten.
Der zweite Weg besteht darin die Datenqualität über mathematische Verfahren festzulegen.
Allerdings werden auch hier häufig Experten benötigt, die die Grenzwerte schätzen.
