\chapter{Experimente}\label{ch:experiments}
%Auf Validität und Reliabilität achten!

Aufbau Experimente: 
Ziele*ʹ Aufbau*ʹ Ergebnisse*' Interpretation*ʹ Threats*to*Validity
(Seite 93 https://userpages.uni-koblenz.de/~laemmel/esecourse/slides/perf.pdf)


Ideen:
- Komplexe Funktionen mit Stakeholdern basteln, zb wenn verheiratet dann Alter > 18
- Daten für zb Aktualität müssen definiert werden, ob sie beispielsweise überhaupt verfallen können. Zb Geburtsdatum ändert sich nie; Alter schon
- Ist es möglich solche Regeln mit HIlfe von ML abzuleiten oder funktioniert das gar nicht? 
- Daten vor einem Monat berechnen, wie viele sich ändern müssten (aufgrund von zb Timeliness, correctness) und dann nachschauen wie viele sich tatsächlich geändert haben
- Big Data Quality A Quality Dimension evaluation hat zwei konkrete Experimente, dort kann man sich gute Ideen holen. Es wird auch ein Experte zu Rate gezogen, der beispielsweise angibt, welche Daten gelöscht werden können (zb wenn 80\% der Attribute fehlen). Es hat auch einige Visualisierungen 
- Mit SQL: 
https://dataform.co/blog/advanced-data-quality-testing

Fragen:
- Woher richtige Daten bekommen?


Auf die verschiedenen Ebenen Aktualität, Richtigkeit, Vollständigkeit und Konsistenz eingehen!
Oft ist es besser die Daten nachzufordern, anhand eines möglichen Fehlers kann nicht der Originalzustand wiederhergestellt werden


- Welche Visualisierungen bieten sich an?
- Gibt es evtl Visualisierungen, die DQ-Probleme aufzeigen?
