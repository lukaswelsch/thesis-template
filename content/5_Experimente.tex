\chapter{Experimente}\label{ch:experiments}
%Auf Validität und Reliabilität achten!
Die ML-Experimente bestehen darin die Methoden praktisch umzusetzen und die Ergebnisse zu interpretieren. 
Ziel der Experimente ist einen Klassifikator zu trainieren, der den Risikoscore zuverlässig vorhersagen kann. 
Mit dessen Hilfe ist es anschließend möglich die Aktualität bzw. Richtigkeit des Scores zu überprüfen und zu aktualisieren.
Dies spart Geld, da nun nur noch die Daten aktualisiert werden müssen, bei denen der Klassifikator eine Abweichung feststellt. 
Auch für neue Kunden kann dieses Verfahren verwendet werden, um schnell ein Ergebnis zu erhalten, da nicht erst auf das Ergebnis der externen Ratingfirma gewartet werden muss.
Allerdings sollten die Daten immer aktualisiert werden und nicht nur auf den Klassifikator vertraut werden. 
Um die Datenmenge zu reduzieren, werden zunächst nur die Daten von einem bestimmten Tag geladen und analysiert.
Da die Klassifikatoren keine guten Ergebnisse aufweisen, wird als Lösung ein Upsampling der Daten vorgeschlagen, das die Lücken der Labels aus älteren Daten füllt.


\section{ML-Experimente / Modell Evaluation }

\subsection{Versuchsaufbau}
Die Experimente werden auf einem Windows-Rechner in einem Jupyter-Notebook durchgeführt. 
%Vorteile von Jupyter Notebook
Als Python-Bibliotheken kommen pandas, numpy und sklearn zum Einsatz. 
\\
Vorgehensweise: \\


Zunächst werden die Daten mit Hilfe eines Python-Connectors von der Datenbank in das Jupyter Notebook geladen, die wie im Kapitel 3 beschrieben, durch ein Skript erzeugt werden.
%Code Exzerpt: 
Hierfür werden die Daten zunächst in ein Python-Dictionary geladen und anschließend in eine Python-Tabelle (DataFrame) gespeichert. 
% was ist ein Python-Dictionary und ein DataFrame 

\subsection{Datenaufbereitung }
Zunächst werden die Daten genauer untersucht um festzustellen ob die Daten bereinigt werden müssen oder ob sich die Klassenverteilung zu sehr unterscheidet. 
Zunächst zeigt eine Grafik die Verteilung der einzelnen Klassen. 
In der Abbildung ist zu erkennen, dass die Klassen sehr ungleichmäßig
verteilt sind.

Klasse 0E ist deutlich häufiger vertreten als manche andere Klassen. 
Dies führt für einige Verfahren zu Problemen und sollte mit Hilfe von Under/Oversampling behoben werden.

Außerdem sind in dem vorliegen Datensatz einige textuelle Attribute, diese können durch einen OneHot Encoder so aufbereitet werden, dass diese in einem Klassifikator verwendet werden können.
Das Feld 'BERUF' besteht aus Berufen, die von den Bankmitarbeitern eingetragen werden, da es nun Berufe gibt, die nicht Genderneutral sind, wird mit Hilfe eines Stemmings die Daten auf eine gemeinsame Grundform überführt.


Da nach einem OneHot Encoding sehr viele Nullen in dem Datenset entstehen bietet sich zur weiteren Verarbeitung eine Sparse-Matrix an.
Diese speichert nur die tatsächlich vorhandenen Daten und speichert keine Null-Werte. %Quelle

Da manche Daten sehr von anderen abweichen (Varianz sehr groß) müssen die Daten skaliert werden.
Zum Beispiel kann der Saldo einen Wertebereich von [-100.000;100.000] haben.
Die Standardabweichung ist sehr groß, da die Daten einen großen Wertebereich haben.
Aus der Formel zur Berechnung der Standardabweichung ergibt sich somit folgende Tabelle:

      | Werte bereich,      | Standardabweichung
SALDO | -100.000, 100.000   | 100.000

%Varianz ausrechnen und als Tabellenausschnitt anzeigen

Des weiteren können Daten entfernt werden, die nur einen einzigen Wert haben, da diese keinen Klassifikator trainieren können.


\subsection{Erzeugen der ABT und Erstellen eines Data Quality Reports}

\subsection{Verfahren}
Die verwendeten Verfahren beruhen auf den in Kapitel 4 ausgewählten Methoden.
Anhand der nachfolgenden Experimente und der genauen Analyse der Daten ist zu Erkennen, dass die Klassenverteilung sehr unausgeglichen ist. 
Dies hat zur Folge, dass die gewählten Modelle zum einen nicht richtig trainiert und zum anderen nicht korrekt validiert werden können. 
Es ist deshalb wichtig die korrekten Verfahren zur Modellbewertung zu beachten.
Das heißt konkret, dass außer dem Accuracy-Score auch die andern Metriken betrachtet werden müssen, wie z.B. Precision und Recall.


\textbf{Warum Undersampling?}
% Eine Baseline stellt ein einfaches Modell dar, das versucht wird zu verbessern. 
Als Beispiel zum Problem der ungleichen Klassenverteilung soll folgendes Szenario zeigen: \\
Die Klassenverteilung der Daten zeigt, dass die Klasse '0E' einen Anteil von knapp 58\% besitzen. 
Ein einfacher Klassifikator, der immer diese Klasse rät, würde somit eine Accurracy von 58\% erreichen. \textbf{HIER NOCH DIE FORMEL EINFÜGEN}
Da Undersampling generell zu weniger Over-Fitting führt als oversampling wird zunächst versucht dieses Verfahren zu verwenden. \cite{fundamentals of ml}
In der Liste der Klassen ist die Klasse '4A' nicht so häufig vertreten, weshalb eine Kombination der beiden Verfahren nachfolgend getestet wird. 
Das Undersampling sollte jedoch nur auf den Trainigsdaten erfolgen und nicht bevor die Daten in Trainings- und Testset extrahiert werden, da sonst die Gefahr besteht, dass das Modell nicht korrekt in der realen Verwendung funktioniert. 

% WIRD JETZT ANDERS GEMACHT Da die Daten nicht in den Arbeitsspeicher des Entwicklungsrechners passen wird die Funktionalität des Train / Testsplit zunächst in SQL geschrieben, damit dies von dem Datenbankserver übernommen werden kann, der genügend Ressourcen zur Verfügung hat. 



Da die Labels der Daten sehr unausgeglichen sind 


%Jedes Verfahren wird mit Upsampling brauchbar gemacht.
%Warum? (grob: was ist upsampling / oder Kapitel 4 verweisen)
%Wie wird das Upsampling gemacht?

KNN
- trainieren
    - Mit / Ohne Upsampling
    - Mit gleicher / ungleicher Klassenverteilung
    - unterschiedliche K - Werte
- Modell Bewerten 
    - train dev test
    - f1-score, accuracy und precision
    - stratified k-fold cross validation
    - f1-score, accuracy und precision 

    
SVM
- trainieren
- Modell Bewerten 
    - train dev test
    - f1-score, accuracy und precision
    - stratified k-fold cross validation
    - f1-score, accuracy und precision 

Random Forest
- trainieren
- Modell Bewerten 
    - train dev test
    - f1-score, accuracy und precision
    - stratified k-fold cross validation
    - f1-score, accuracy und precision 


Zusammenfassung: Welcher hat das beste Klassifikationsergebnis?
In diesem Kapitel wurden die Verfahren KNN, SVM, blabla getestet und der beste Klassifikator ist blabla.






%Scoringverfahren:



%Was sind Daten die Einfluss auf das Kundenrating haben?
%-> Für welche Daten wäre es besonders wichtig die Datenqualität händisch zu überprüfen


%ML-Verfahren, immer inklusive Performance-Grafik:
%- logistische Regression


%Ziele:
%- Vorhersagen des Risikoscorings anhand der Inputdaten

%Aufbau:
%- python als Programmiersprache
%- Aufteilen in train-dev-test
%- 


%\subsection{KNN}

%Variabilität (Einfluss der Parameter auf das Ergebnis):
%- verschiedene k-Werte 
%- Manhattan vs. Euklidische Distanz


%\subsection{SVM (Multiklassen)}

%Verschiedene Parameter:
%- unterschiedl. Kernelfunktionen
%- C-Parameter, der über Slackvariablen gesteuert wird
%- 

%- Neuronale Netze

%- Classifikation Tree



%Aufbau Experimente: 
%Ziele*ʹ Aufbau*ʹ Ergebnisse*' Interpretation*ʹ Threats*to*Validity
%(Seite 93 https://userpages.uni-koblenz.de/~laemmel/esecourse/slides/perf.pdf)


%Ideen:
%- Komplexe Funktionen mit Stakeholdern basteln, zb wenn verheiratet dann Alter > 18
%- Daten für zb Aktualität müssen definiert werden, ob sie beispielsweise überhaupt verfallen können. Zb Geburtsdatum ändert sich nie; Alter schon
%- Ist es möglich solche Regeln mit HIlfe von ML abzuleiten oder funktioniert das gar nicht? 
%- Daten vor einem Monat berechnen, wie viele sich ändern müssten (aufgrund von zb Timeliness, correctness) und dann nachschauen wie viele sich tatsächlich geändert haben
%- Big Data Quality A Quality Dimension evaluation hat zwei konkrete Experimente, dort kann man sich gute Ideen holen. Es wird auch ein Experte zu Rate gezogen, der beispielsweise angibt, welche Daten gelöscht werden können (zb wenn 80\% der Attribute fehlen). Es hat auch einige Visualisierungen 
%- Mit SQL: 
%https://dataform.co/blog/advanced-data-quality-testing




%Auf die verschiedenen Ebenen Aktualität, Richtigkeit, Vollständigkeit und Konsistenz eingehen!
%Oft ist es besser die Daten nachzufordern, anhand eines möglichen Fehlers kann nicht der Originalzustand wiederhergestellt werden

%\section{Visualisierung}

%https://www.elastic.co/guide/en/kibana/6.8/createvis.html

\subsection{Visualisierung}
Damit die Daten für die Visualisierung verfügbar sind, müssen diese zunächst in das Visualisierungstool integriert werden. 
Da zur Erstellung des Dashboards das Tool Grafana vorgeschlagen und verwendet wird, müssen die Daten an Grafana angebunden werden.
Hierfür wäre es sinnvoll einen Konnektor zu verwenden, der es ermöglicht Grafana direkt mit der DB2 Datenbank zu verbinden. 
Es sind einige Konnektoren als Plugin für Grafana verfügbar, allerdings fehlt dort die Möglichkeit eine DB2-Datenbank anzubinden. \cite{https://grafana.com/grafana/plugins?type=datasource}
Auch fehlt die Möglichkeit über eine allgemeine JDBC-Verbindung proprietäre Datenbanksysteme anzubinden. 
Die Daten werden mit Hilfe der DB2-Exportfunktionalität exportiert und als CSV-Datei gespeichert. 
Diese Datei wird anschließend mit Hilfe von SQL in eine MySQL Datenbank exportiert. 
Um die Daten regelmäßig zu migrieren kann ein Cron-Job verwendet werden, der beispielsweise einmal in der Nacht ausgeführt wird. 
Als Alternative zur Verwendung der MySQL-Datenbank ist das Plugin CSV von Grafana zu nennen. 
Allerdings ist dieses Plugin noch in einer frühen Entwicklungsphase und wird deshalb nicht in diesem Projekt verwendet. \cite{https://grafana.com/grafana/plugins/marcusolsson-csv-datasource?pg=plugins&plcmt=featured-undefined}
Des weiteren bietet Grafana die Möglichkeit Daten mit Hilfe eines JSON-Formats zu importieren. 
Da es allerdings leichter ist Daten mit Hilfe eines SQL-Syntaxes abzufragen wird hier der Weg gewählt die Daten mit Hilfe der MySQL-Datenbank anzubinden.


- Data Quality Report anhand der ABT.


Integration der Daten in Kibana
- ETL Prozess zur automatischen Erzeugung einer CSV-Datei.
- Logstash?
- 

- Visualisierung von Unterschieden:
- Aggregiert nach Wochentag / Monat
- Umsatzdaten
- Kundendaten (Neukundenzuwachs?)


Kibana, Graphana

Reduktion der Datenmenge



- Welche Visualisierungen bieten sich an?
- Gibt es evtl. Visualisierungen, die DQ-Probleme aufzeigen?


