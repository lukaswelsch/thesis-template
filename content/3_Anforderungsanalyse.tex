\chapter{Anforderungsanalyse}\label{ch:anforderungen}
Um aus dem Data Warehouse die geeigneten Daten zu extrahieren und Methoden zur Analyse und Visualisierung der Datenqualität auszuwählen, ist es zunächst notwendig eine Stakeholder-Analyse durchzuführen. 
Auf Basis dieser Analyse ist es anschließen möglich die relevanten Daten aus dem Data Warehouse zu ermitteln, die für die Durchführung der Methoden notwendig sind. 



Für eine Datenqualität, die sich in einem Unternehmen etablieren kann, ist es nötig diese von den Gesichtspunkten aller Stakeholdern zu betrachten.
Anhand einer Stakeholder-Analyse wird es möglich sein die Probleme der Datenqualität auf zwei Ebenen zu betrachten. 
Zum einen benötigen die Business-User gute Datenqualität, um die richtigen Zinsen anhand des Risikoscorings zu vergeben und auf der anderen Seite benötigen die Entwickler eine Möglichkeit, um ihre ETL-Strecken zu überprüfen.
Für diese beiden Probleme können zwei verschiedene Verfahren eingesetzt werden.
Zur Identifikation und Überprüfung des Risikoscores kann ein Machine Learning Algorithmus trainiert werden, der die korrekten Werte vorhersagt und somit einen Aufschluss darüber gibt, wie viele Datensätze aktualisiert werden sollten.
Um die ETL-Strecken zu analysieren sind die Metadaten zu den Prozessen nötig, die anzeigen, wie viele Daten an jedem Tag extrahiert wurden. 
Auffälligkeiten können dann genutzt werden, um Maßnahmen einzuleiten.
Hierfür bietet sich eine Visualisierung an, die in Echtzeit die Daten verarbeitet, analysiert und anschließend interpretierbar darstellt.
Die beiden beschriebenen Verfahren werden in dem Kapitel Experimente durchgeführt. 
%Datakitchen



\section{Stakeholder-Analyse}
\label{sec:Stakeholder-Analyse} 
Mit Hilfe einer Stakeholder-Analyse ist es möglich, die negativen Einflüsse zu erkennen und die positiven Einflüsse zu nutzen.
Zum Anderen können die Erwartungen der einzelnen Stakeholder erkannt und die Projektziele richtig gewichtet werden. \\
Die Stakeholder-Analyse wird nach folgenden Schritten durchgeführt: 

\begin{enumerate}
 \item Stakeholder identifizieren
 \item Stakeholder Einfluss analysieren
 \item Aktionsplanung (Maßnahmen ableiten)
\end{enumerate} \cite{salzsearch}

% Nach \cite{pipino2002} gibt es drei Stakeholder: the collector (Sammler) , custodians (Verwalter) and consumers (Verbraucher) of data products.  

\subsection{Stakeholder identifizieren}
In diesem Schritt werden die Stakeholder genannt, kurz beschrieben und visualisiert. 
Aufgrund der Analyse des mittelständischen Unternehmens aus der Bankenbranche können folgende Stakeholder identifiziert werden: Entwickler, Business User und Bankmitarbeiter.
Diese werden im Folgenden genauer analysiert und beschrieben. 


\begin{itemize}
% \itemsep-2em 
\item \textbf{Entwickler} \\            
Als Stakeholder lassen sich zunächst die Entwickler identifizieren.
Diese sind für die Einhaltung einer guten Datenqualität unmittelbar wichtig, da diese die Extraktionsprozesse entwickeln, die zu Fehlern in der Datenqualität führen können.
%Warum kann es überhaupt passieren, dass nicht alle Daten exportiert werden?
Die Daten werden von einem legacy-System in das Data Warehouse übertragen, indem die Daten aus einer hierarchischen Datenbank als Datei abgelegt werden.
Ein Prozess iteriert über alle Dateien und extrahiert die Daten, die anschließend neu strukturiert im Data Warehouse abgelegt werden.
Aufgrund von einigen Problemen in der Extraktion kann es dazu führen, dass die Daten nicht korrekt extrahiert werden.
Für diesen Fall gibt es eine Überwachung der Programme, die einen Fehler bei der Verarbeitung meldet.
Wenn jedoch weniger oder mehr Daten verarbeitet werden als üblich deutet dies auch auf einen Fehler in der Extraktion hin. 
Dieser Fall wird aktuell noch durch keinen Mechanismus oder Programm erkannt. 
Als Lösung für dieses Problem bietet sich eine Visualisierung an, anhand der die Schwankungen aufgezeigt und dargestellt werden können. 
Ein Entwickler kann anhand der Visualisierung erkennen, ob durch seine Programmänderung die Schwankung der Datenmengen innerhalb der saisonalen Schwankungen ist oder, ob seine Änderung einen neuen Fehler verursacht hat. 
%Probleme: Keine Datei wird mehr an einem Ordner abgelegt, weil fehler im anderen System -> keiner merkt was, Daten nicht mehr aktuell
%Probleme: Zweite Änderung während Extraktion läuft -> Primärschlüssel doppelt -> deduplizierung -> wenn fehler in Primkey, dann werden ganz viele Daten gelöscht, was ungewöhnlich ist
%Allerdings werden die Dateien teilweise nicht schnell genug abgeholt und die Daten werden überschrieben


\item \textbf{Business User}       \\
Auf der anderen Seite gibt es die Business User, die beispielsweise die Gesamtbanksteuerung übernehmen.
Bei diesem Teilgebiet wird errechnet, wie gut die Zinssätze sein dürfen, sodass die Bank Gewinn erzielt und gleichzeitig möglichst viele Kunden zufrieden stellen kann.
Für diesen Stakeholder sind alle Dimensionen wichtig, da nur mit einer sehr guten Datenqualität eine optimale Gesamtbanksteuerung vorgenommen werden kann. 
Würde der Stakeholder nur über falsche, unvollständige und veraltete Daten verfügen, könnte dies Verluste für die Bank zur Folge haben.
Besonders entscheidend ist für diesen Stakeholder, dass der Wert für den Risikoscore richtig und aktuell ist.
Beim Risikoscoring werden die Daten der Kunden an einen externen Dienstleister gesendet.
Dieser berechnet einen Risikoscore, anhand dessen die Zinsvergabe erfolgt.
%CRM sind auch Business User, die korrekte Daten haben wollen

\item \textbf{Bankmitarbeiter}      \\
Des weiteren ist der Bankmitarbeiter Bestandteil der Kreditvergabe. 
Die Kreditvergabe erfolgt anhand einer Bewertung des Kunden, je nachdem wie gut der Risikoscore des Kunden ist, umso bessere Zinssätze erhält dieser.
Für den Bankmitarbeiter ist es wichtig, dass die Daten zum Risikoscoring sowohl richtig als auch aktuell sind.

\end{itemize}




Nicht nur Fehler in numerischen Daten, sondern auch in Texten sind festzustellen. 
Dies führt zu falschen Interpretationen bzw. dazu, dass ein Kunde nicht den richtigen Risikoscore erhält. 
Auch fehlende Daten führen zu Fehlern.
Dies ist ein Fall, der sowohl für den Kunden als auch für die Bank selbst zu einem Problem führt. 
Der Kunde erhält nicht den für ihn bestmöglichen Risikoscore und die Bank verliert einen Kunden, da diese verglichen mit der Konkurrenz nicht den besten Kredit vergeben kann. 
Es kann allerdings auch passieren, dass die Bank dem Kunden einen besseren Score vergibt, da der Beruf falsch interpretiert wird und muss somit ein höheres Risiko eingehen und die Bank kann ihr Risiko nicht korrekt abschätzen. 






Die beschriebenen Stakeholder werden mit ihren Erwartungen an das Projekt in der folgenden Grafik dargestellt.
\subsection{Betroffenheitsanalyse}
\begin{tabular}[h]{l|p{2cm}|>{\centering}p{1.5cm}|p{2.5cm}|c|c|p{3cm}}
ID & Wer        & Betroffen-heit & Erwartung & Macht & Einstellung & Maßnahmen  \\ \hline
S1 & Entwickler & m             & Fehler in der Entwicklung werden angezeigt; Wenig Programmieraufwand & g & Neutral & Erklärung der Notwendigkeit, Zeitvorteil aufzeigen  \\ \hline
S2 & Business User & h          & Daten sollten fehlerfrei und aktuell sein & g & Positiv & - \\ \hline
S3 & Bank-mitarbeiter & h          & Daten müssen immer verfügbar und richtig sein & h & Positiv & - \\
\end{tabular}


Welchen Stakeholder interessiert welche Dimension?

\begin{tabular}[h]{l|l|l}
Stakeholder & Dimension & Begründung \\ \hline
Entwickler & Vollständigkeit & Der Entwickler geht davon aus, dass die Daten die vom Quellsystem abgeholt werden richtig sind. Für ihn ist deshalb die Dimension Vollständigkeit interessant, da er wissen muss, ob die richtige Anzahl an Daten exportiert wurden. \\ \hline
Business User & Alle Dimensionen & \\ \hline
Bankmitarbeiter & Keine Dimension & Geht davon aus, dass die Datenqualität schon geprüft wurde \\
\end{tabular}


Für Welche Stakeholder kann sollte welches Verfahren angewendet werden?
- Business User: Risikoscoring sollte immer aktuell sein, dabei kann ein ML Verfahren verwendet werden, um die Daten zu klassifizieren und bei großer Abweichung können diese Daten dann aktualisiert oder neu angefordert werden.

- Bank Mitarbeiter: ?
- 

In dieser Bachelorarbeit wird die Verwendung von Machine Learning Verfahren zur Verbesserung der Datenqualität untersucht.
Deshalb bietet sich das Thema Risikoscoring am besten an, da es feste Ausgangswerte (Labels) bietet, die trainiert werden können.
Der trainierte Klassifikator kann anschließend verwendet werden, um zu berechnen, ob ältere Risikoscorings vom Ergebnis des Klassifikators abweichen.
Denn es können sich Attribute aktualisieren, die einen positiven oder negativen Einfluss des Risikoscorings zur Folge haben. 
Allerdings wird das Risikoscoring nicht regelmäßig aktualisiert, da eine Aktualisierung Geld kostet.
Grundvoraussetzung für dieses Vorhaben ist die Risikoscorings anhand von aktuellen Daten mit hoher \colorbox{green}{Präzision} vorherzusagen. 
Dafür werden nachfolgend Klassifikatoren ausgewählt und anschließend im Kapitel Experimente geprüft und ausgewertet.\\
Allerdings sollten auch für die anderen Stakeholder Verfahren entwickelt werden, um die Datenqualität zu verbessern.
Dies beinhaltet unter anderem eine Peer-Review, um die Codequalität sicherzustellen, sowie statische Code-Analysen.
Eine bessere Codequalität führt zu einer besseren Datenqualität, da keine bzw. weniger menschliche Fehler im System erzeugt werden.
Des weiteren ist es notwendig die Datenbankschemas gut und exakt zu definieren, sodass keine null-Werte an Stellen eingefügt werden, an denen es keine null-Werte geben darf.
Diese Lücken in den Daten sollten immer mit einem Default-Wert belegt werden und null-Values zu einem Abbruch der ETL-Strecken führen.
Dadurch kann die Dimension der Vollständigkeit verbessert werden.
Auch Log-Dateien die Aussagen über den ETL-Prozess liefern, können hilfreich sein und sollten ausgewertet werden.
\colorbox{green}{Aufzeigen, warum es wichtig ist auch Visualisierungen zu verwenden, um Datenqualität zu analysieren}

\subsection{Anforderungen an den Klassifikator}
Damit der trainierte Klassifikator verwendet werden kann, um weniger Risikoswerte bei dem externen Ratingfirma abzufragen, muss der Klassifikator einige Anforderungen erfüllen.
Diese können als Fragen gestellt werden, die anschließend mit Hilfe der ML-Experimente beantwortet werden können. 
Diese Fragen umfassen auch die zuvor aufgestellten Thesen, zur Verwendung von bestimmten Datensätzen als zusätzliche Verbesserung der Klassifikation. 

\begin{itemize}
 \item Welchen Einfluss haben die zusätzlich ausgewählten Eigenschaften auf die Güte?
 \item Wie hoch ist insbesondere die Precision des Klassifikators?
 \item Welcher Klassifikator ist am besten geeignet und erzielt den höchsten Precisionwert?
 \item Können auch ohne die Daten vom externen Anbieter Schufa gute Ergebnisse erzielt werden?
 \item Was eignet sich besser vorherzusagen der Schufascore oder Ratingwert?
\end{itemize}



\section{(Statistsiches Verfahren) ABT / Voranalyse}
Mit Hilfe einer sogenantnen Analytical Base Table ist es möglich erste Aussagen über die Datenqualität der zu prüfenden Daten zu treffen.
Zunächst wird diese generiert, um festzustellen, ob die Daten für weitere Untersuchen verwendet werden können.
Anschließend kann ein Plan entwickelt werden, der darstellt, welche Maßnahmen bei unterschiedlichen Datenqualitätsproblemen getroffen werden kann.

