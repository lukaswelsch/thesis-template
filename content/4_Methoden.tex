\chapter{Methoden}\label{ch:method}
Um die nachfolgenden Methoden zur Analyse und Visualisierung von Datenqualität, zu vergleichen werden sowohl qualitative als auch quantitative Untersuchungen durchgeführt.

Nach \cite{pipino2002} besteht die Schwierigkeit nicht darin die Metriken zu formulieren, sondern die Datenqualitätsdimension zu definieren, die auf den spezifischen Anwendungsbereich des Unternehmens passt. 

% (In Worten zusammengefasst und durch Statistiken und Schaubilder) 
Metriken:


\section{Metriken}
Simple Ratio. 1 - Unvollständig / Vollständig
Desired outcomes to total outcomes 
-> geeignet für Richtigkeit, Vollständgkeit, Konsistenz


Maximum operation
-> geeignet für Aktualität

Weighted Average. 


\section{Bestehende Verfahren}
Richtigkeit
Um die Richtigkeit zu berechnen gibt es zwei verschiede Möglichkeiten. 


\section{Stakeholderanalyse}
Nach \cite{pipino2002} gibt es drei Stakeholder: the collector, custodians and consumers of data products

Es besteht die Möglichkeit durch einen Fragebogen die vorherrschende Datenqualität mit Hilfe der Stakeholder zu berechnen. \cite{pipino2002}


In this chapter, we're actually using some code!

\begin{lstlisting}[language=Python,caption={This is an example of inline listing},captionpos=b]
x = 1
if x == 1:
    # indented four spaces
    print("x is 1.")

\end{lstlisting}

You can also include listings from a file directly:

\lstinputlisting[language=Python,caption={This is an example of included listing},captionpos=b]{listings/example.py}
